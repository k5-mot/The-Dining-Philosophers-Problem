\pagestyle{plain}
%
%% Package ------------------------------------
\usepackage[dvipdfmx]{graphicx}
\usepackage[dvipdfmx]{color}
\usepackage{amsmath,amssymb}
\usepackage{ascmac}
\usepackage{bm}
\usepackage{multicol}
\usepackage{titlesec}
\usepackage{listings,style/jlisting}
\usepackage{url}
%
%% Package Settings ---------------------------
\setlength{\textwidth}{\fullwidth}
\setlength{\textheight}{39\baselineskip}
\addtolength{\textheight}{\topskip}
\setlength{\voffset}{-0.5in}
\setlength{\headsep}{0.3in}
\setlength{\topmargin}{0pt}
\setlength{\headheight}{0pt}
%
% 限界まで余白を削る
\setlength{\hoffset}{-0.5in}
\addtolength{\textwidth}{1.0in}
%
\definecolor{deepblue}{rgb}{0,0,0.5}
\definecolor{deepred}{rgb}{0.6,0,0}
\definecolor{deepgreen}{rgb}{0,0.5,0}
%
%% Code printing settings.
\lstset{
  basicstyle={\ttfamily},
  identifierstyle={\small},
  %commentstyle={\small\itshape\color{deepgreen}},
  commentstyle={\small\color{deepgreen}},
  keywordstyle={\small\bfseries\color{deepblue}},
  ndkeywordstyle={\small},
  stringstyle={\small\ttfamily\color{deepred}},
  tabsize=4,
  frame={tb},
  breaklines=true,
  columns=[l]{fullflexible},
  numbers=left,
  xrightmargin=3zw,
  xleftmargin=3zw,
  numberstyle={\scriptsize},
  stepnumber=1,
  numbersep=1zw,
  lineskip=-0.5ex
}
%
\titleformat*{\section}{\Large\bfseries}
\titleformat*{\subsection}{\normalsize\bfseries}
\renewcommand{\descriptionlabel}[1]{\hspace{\labelsep}\textbf{#1}}
%
% Use divergence
\newcommand{\divergence}{\mathrm{div}\,}
% Use gradient
\newcommand{\grad}{\mathrm{grad}\,}
% Use rotation
\newcommand{\rot}{\mathrm{rot}\,}
% Use date for Japanese
\newcommand{\todayJP}{\number\year 年\number\month 月\number\day 日}
%
% the caption label for it.
\renewcommand{\figurename}{図}
\renewcommand{\tablename}{表}
\renewcommand{\lstlistingname}{コード}
% the header name for the list of it.
\renewcommand{\listfigurename}{図目次}
\renewcommand{\listtablename}{表目次}
\renewcommand{\lstlistlistingname}{コード目次}
%
%
%% \maketitleの余白を調整 ver 1.0
% \makeatletter
% \renewcommand{\@maketitle}{\newpage
% % \null
% % \vskip 2em
% \begin{center}
% {\LARGE \@title \par} \vskip 1.5em {\large \lineskip .5em
% \begin{tabular}[t]{c}\@author
% \end{tabular}\par}
% \vskip 1em {\large \@date} \end{center}
% \par
% \vskip 1.5em}
% \makeatother
%
%% \maketitleの余白を調整 ver 2.0
\makeatletter
\def\subject#1{\def\@subject{#1}}
\def\id#1{\def\@id{#1}}
\def\master#1{\def\@master{#1}}
\def\department#1{\def\@department{#1}}
\def\@maketitle{
\begin{center}
{\Large \@subject \par} %修士論文と記載される部分
%\vspace{10mm}
{\huge\bf \@title \par}% 論文のタイトル部分
\vspace{5mm}
{\large \@department \@master \par} % 所属部分
{\large \@id \@author\par} % 学籍番号部分
\vspace{5mm}
\end{center}
%\par\vskip 1.5em
}
\makeatother
%
%
% 折り返し
\sloppy
