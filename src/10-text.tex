\section{元のコード}

問題に沿って、モデルを記述する。
コード~\ref{code:original}内の philosopher プロセスは、
哲学者の行動を表している。

\lstinputlisting[language={promela},
	caption={元のコード (dining\_philosophers\_v1.pml)},
	label={code:original}]
	{code/dining_philosophers_v1.pml}

\newpage

\section{デッドロックの回避}

コード~\ref{code:original}では、
全ての哲学者が右手にフォークを持っていた場合、
全ての哲学者が左側にフォークが置かれるのを待つ状態になり、
これ以上プロセスが進まないデッドロックに陥る。

このデッドロックを回避するために、哲学者は、
\begin{enumerate}
	\item 右側のフォークを取る
	\item 左側のフォークが既に取られていた場合、右手のフォークを置く
\end{enumerate}
という操作を行う(コード\ref{code:deadlock})。

\lstinputlisting[language={promela},
	caption={デッドロックを回避するコード (dining\_philosophers\_v2.pml)},
	label={code:deadlock}]{code/dining_philosophers_v2.pml}

\newpage

\section{進行性の成立}

コード~\ref{code:deadlock}では、
ある哲学者が連続して右手にフォークを持ち続けた場合、
その右の哲学者は左手にフォークを持つことができないため、
進行性が成立しない。

進行性を成立させるために、哲学者の食事の管理を行う 
管理者を用意した。
管理者は哲学者の食事要請をキューに格納し、
順番に一人ずつ哲学者に食事をさせる。

管理者と哲学者は、
\begin{enumerate}
	\item 哲学者が管理者に食事要請メッセージを送る
	\item 管理者は食事要請メッセージをキューに入れる
	\item 管理者はキュー先頭の食事要請メッセージに対して、食事開始メッセージを送る
	\item 哲学者は食事開始メッセージを受ける
	\item 哲学者は右・左のフォークを持つ
	\item 哲学者は食事する
	\item 哲学者は右・左のフォークを置く
	\item 哲学者は管理者に食事終了メッセージを送る
	\item 管理者は食事終了メッセージを受ける
	\item 3. へ戻り、キューの食事要請を処理する
\end{enumerate}
の流れでやりとりを行う(コード\ref{code:progressive})。
コード~\ref{code:progressive}の manager プロセスは、
管理者の行動を表している。

\lstinputlisting[language={promela},
	caption={進行性が成立するコード (dining\_philosophers\_v3.pml)},
	label={code:progressive}]
	{code/dining_philosophers_v3.pml}
